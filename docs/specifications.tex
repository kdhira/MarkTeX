\documentclass[10pt, a4paper]{article}

\title{MarkTeX}
\author{Kevin Hira}

\begin{document}
\maketitle

\section*{Introduction}
MarkTeX is an application that converts markup (markdown) into \LaTeX{} code. This converter application will allow files written in markdown (and even with raw \LaTeX{} code mixed in) to be converted to \LaTeX{} documents that can be compiled to the Portable Document Format (pdf) without error. MarkTeX does not need to comply fully with other markup languages entirely; the markup language that is used with MarkTeX will most likey be an off-shoot of markdown.

\pagebreak

\section{Requirements}
\subsection{Markup Parsing}
MarkTeX should be able to parse the general markup rules, like those in the markdown langauge, namely:
\begin{itemize}
    \item Headings
    \begin{itemize}
        \item \verb!# Heading! should convert to \verb!\section{Heading}!.
        \item \verb!## Subheading! should convert to \verb!\subsection{Subheading}!.
        \item \verb!### Subsubheading! should convert to \newline \verb!\subsubsection{Subsubheading}!.
    \end{itemize}
    \item Paragraphs and Formatting
    \begin{itemize}
        \item Paragraphs should be separated by blank lines.
        \item Lines ending in two spaces should create a line break in paragraphs, \verb!\\! or \verb!\newline!.
        \item \verb!_text_! and \verb!*text*! should convert to \verb!\textit{text}!.
        \item \verb!__text__! and \verb!**text**! should convert to \verb!\textbf{text}!.
        \item \texttt{\`}\verb!text!\texttt{\`} should convert to \verb!\texttt{text}! (or even perhaps \newline \verb+\verb!text!+).
        \item \verb!---! should convert to \verb!\hline!.
    \end{itemize}
    \item Links
    \begin{itemize}
        \item \verb![link-url]! should convert to \verb!\url{link-url}!.
        \item \verb![link-text](link-url)! should convert to \newline \verb!\href{link-url}{link-text}!.
        \item URLs that are written without any wrapping should attempt to be parse to links.
        \item Links should also work with mailto links.
    \end{itemize}
    \item Lists
    \begin{itemize}
        \item Lines that start with \verb!-! should add to a unordered list with the \textit{itemize} environment.
        \item Lines that start with contiguous numbers followed by periods should add to a ordered list with the \textit{enumerate} environment.
        \item Nested lists should be able to be parsed correctly.
    \end{itemize}
    \item Code
    \begin{itemize}
        \item Code surrounded by \texttt{\`}\texttt{\`}\texttt{\`} should be handled appropriately.
    \end{itemize}
\end{itemize}

\subsection{\LaTeX{} Preserving}
MarkTeX should also support raw \LaTeX{} in documents, and upon rendering/par\-sing, all \LaTeX{} code should be preserved.

\subsection{Document Layout}
MarkTeX documents should have a certain layout, which consists of a plain text document starting with the compulsory front matter secton followed by markdown/\LaTeX{} code (the body of the document).
\subsubsection{Front Matter}
A MarkTeX document should start with what is known as front matter. The front matter is a header section that defines variables that can be used or refered to throughout the document. This section should start with a line containing only \verb!---! and ending on a line containing only \verb!---!. Variable declarations are one per line and the name and value are separated by a colon (:), eg \verb!title: MarkTeX!
\subsubsection{Example/Template}
\begin{verbatim}
---
title: Specifications
author: Kevin Hira
date: 11 December 2016
a: 1
b: true
---
# Hello World
## A Template
Hello World
\end{verbatim}

\subsection{Generating \LaTeX{}}
MarkTeX, upon execution of the program, should generate a error free (when possible) \LaTeX{} document that can later be used to generate a pdf output from.

\end{document}
